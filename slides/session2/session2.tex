\documentclass[10pt]{beamer}

\usetheme[progressbar=frametitle]{metropolis}
\usepackage{appendixnumberbeamer}
\usepackage[utf8]{inputenc}
\usepackage{booktabs}
\usepackage[scale=2]{ccicons}

\usepackage{pgfplots}
\usepgfplotslibrary{dateplot}

\usepackage{xspace}
\newcommand{\themename}{\textbf{\textsc{metropolis}}\xspace}

\title{KodeKlubben 2.0}
\subtitle{Øvelsesgang 1}
% \date{\today}
\date{\today}
\author{Kristian Urup Olesen Larsen}
\institute{Økonomisk Institut, KU}
% \titlegraphic{\hfill\includegraphics[height=1.5cm]{logo.pdf}}

\begin{document}

\maketitle

\begin{frame}[fragile]{Velkommen!}
Hvem er vi?
\begin{itemize}
  \item Økonomistuderende
\end{itemize}
Hvem er i?
\begin{itemize}
  \item Det er sådan set ligemeget, vi skal bare have det sjovt og lære noget \textit{data science}.
\end{itemize}
\end{frame}

\begin{frame}[fragile]{Hvad skal vi lave?}
Vi skal lave et \textit{mini}projekt: dataindsamling, tidying og til sidst en pæn figur.
\begin{itemize}
  \item Et sæt exercises (\textit{krykker}) til hver session.

\end{itemize}

\end{frame}

\begin{frame}[fragile]{Projektet}

Denne gang:
\begin{itemize}
  \item
\end{itemize}

\end{frame}


\end{document}

\documentclass[10pt, hyperref = {colorlinks=true, linkcolor=green}]{beamer}

\usetheme[progressbar=frametitle]{metropolis}
\usepackage{appendixnumberbeamer}
\usepackage[utf8]{inputenc}
\usepackage{booktabs}
\usepackage[scale=2]{ccicons}

\usepackage{pgfplots}
\usepgfplotslibrary{dateplot}

\usepackage{xspace}
\newcommand{\themename}{\textbf{\textsc{metropolis}}\xspace}

\title{KodeKlubben 2.0}
\subtitle{Øvelsesgang 1}
% \date{\today}
\date{\today}
\author{Kristian Urup Olesen Larsen}
\institute{Økonomisk Institut, KU}
% \titlegraphic{\hfill\includegraphics[height=1.5cm]{logo.pdf}}

\begin{document}

\maketitle

\begin{frame}[fragile]{Velkommen!}
Hvem er vi?
\begin{itemize}
  \item Økonomistuderende
  \item RA's på Økonomisk Institut
  \item Arbejder typisk i Python, R, STATA eller SAS
\end{itemize}
Hvem er i?
\begin{itemize}
  \item Det er sådan set ligemeget, vi skal bare have det sjovt og lære noget \textit{data science}.
\end{itemize}
\end{frame}

\begin{frame}[fragile]{Hvad skal vi lave?}
\begin{itemize}
\item Vi skal lave et \textit{mini}projekt: dataindsamling, tidying og til sidst en pæn interaktiv figur.
\item Vi kan ikke lære jer at skrive kode, men vi kan hjælpe jer med at øve
\begin{itemize}
  \item Et sæt exercises til hver session.
  \item Slides så vi kan gennemgå øvelserne sammen.
\end{itemize}
\end{itemize}


\metroset{block=fill}
\begin{block}{Dokumentation}
Man \textit{kan} ikke vide alt om Python, derfor er (næsten) alt funktionalitet beskrevet i søgbar, overskuelig, præcis (og teknisk) dokumentation. \textbf{Google} er altid den hurtigste vej til at finde denne, kode-eksempler, usecases osv.
\end{block}

\end{frame}

\begin{frame}[fragile]{Projektet}
Op til KV17 lavede DR en \href{https://www.dr.dk/nyheder/politik/kv17/kandidat-testen}{kandidattest}. Alle kandidaterne har selvfølgelig svaret på spørgsmålene. De svar vil vi gerne have.

Denne gang:
\begin{itemize}
  \item
\end{itemize}

\end{frame}


\end{document}

\documentclass[10pt]{beamer}

\usetheme[progressbar=frametitle]{metropolis}
\usepackage{appendixnumberbeamer}
\usepackage[utf8]{inputenc}
\usepackage{booktabs}
\usepackage[scale=2]{ccicons}
\usepackage{xcolor}
\usepackage{pgfplots}
\usepgfplotslibrary{dateplot}

\usepackage{xspace}
\newcommand{\themename}{\textbf{\textsc{metropolis}}\xspace}

\title{\texttt{pip}}
% \date{\today}
\date{\today}
\author{Kristian Urup Olesen Larsen, Jakob Jul Elben}
\institute{Økonomisk Institut, KU}
% \titlegraphic{\hfill\includegraphics[height=1.5cm]{logo.pdf}}

\begin{document}

\maketitle

\begin{frame}[fragile]{Pip}
  Python er en basispakke, men tusindvis af mennesker har udviklet \textit{moduler} som udvider funktionaliteten. De installeres med \texttt{pip}.

How to:
\begin{itemize}
  \item[1.] Åben en terminal (mac)/kommandoprompt (windows)
  \item[2.] skriv: \texttt{pip install \textit{modulnavn}} og tryk \texttt{\textsc{enter}}
\end{itemize}

\end{frame}


\end{document}
